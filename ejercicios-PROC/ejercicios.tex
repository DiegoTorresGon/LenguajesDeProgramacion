\documentclass[12pt]{article}

\usepackage[shortlabels]{enumitem}

\begin{document}

\title{Ejercicios PROC}
\author{Diego Torres Gonzalez}
\date{\today}

\maketitle

\begin{itemize}

\item[3.20]{In PROC, procedures have only one argument, but one can get the effect of multiple argument procedures by using procedures that return other procedures. 
For example, one might write code like}

	let f = proc (x) proc (y) ... \\
	in ((f 3) 4) \\

	This trick is called Currying, and the procedure is said to be Curried. Write a Curried procedure that takes two arguments and returns their sum. 
	You can write $x + y$ in your language by writing $-(x, -(0, y))$. \\

	La funci\'on quedar\'ia as\'i:\\
		proc (x) proc (y) $-(x, -(0, y))$

\item[3.27]{Add a new kind of procedure called a traceproc to the language. A traceproc works exactly like a proc, except that it prints a trace message on
entry and on exit.}

	Expression ::= traceproc (Identifier) Expression \\\\
	traceproc (symb) expr ::= proc (symb) (print "Enter proc with val: " symb " and body: " expr) expr (print "Exit proc with val: " symb)\\\\



\end{itemize}


\end{document}

